\documentclass{article}

% if you need to pass options to natbib, use, e.g.:
%     \PassOptionsToPackage{numbers, compress}{natbib}
% before loading neurips_2018

% ready for submission
% \usepackage{neurips_2018}

% to compile a preprint version, e.g., for submission to arXiv, add add the
% [preprint] option:
%     \usepackage[preprint]{neurips_2018}

% to compile a camera-ready version, add the [final] option, e.g.:
     \usepackage[final, nonatbib]{nips_2018}

% to avoid loading the natbib package, add option nonatbib:
%     \usepackage[nonatbib]{neurips_2018}

\usepackage[utf8]{inputenc} % allow utf-8 input
%\usepackage[cp1251]{inputenc}
\usepackage[T1]{fontenc}    % use 8-bit T1 fonts
\usepackage{hyperref}       % hyperlinks
\usepackage{url}            % simple URL typesetting
\usepackage{booktabs}       % professional-quality tables
\usepackage{amsfonts}       % blackboard math symbols
\usepackage{nicefrac}       % compact symbols for 1/2, etc.
\usepackage{microtype}      % microtypography
\usepackage[main=russian, english]{babel}
\usepackage{graphicx}
\usepackage{float}
\usepackage{amsfonts,amsmath,amssymb}
%\usepackage{natbib}

\title{Отчет по воспроизведению статьи "Generative Modeling by Estimating Gradients of the Data Distribution"}

% The \author macro works with any number of authors. There are two commands
% used to separate the names and addresses of multiple authors: \And and \AND.
%
% Using \And between authors leaves it to LaTeX to determine where to break the
% lines. Using \AND forces a line break at that point. So, if LaTeX puts 3 of 4
% authors names on the first line, and the last on the second line, try using
% \AND instead of \And before the third author name.

\author{%
  Петр Жижин \\
  Факультет Компьютерных наук\\
  ВШЭ\\
  \texttt{??} \\
  % examples of more authors
   \and
   Даяна Савостьянова \\
   Факультет Компьютерных наук\\
   ВШЭ\\
   \texttt{dayanamuha@gmail.com} \\
  % \AND
  % Coauthor \\
  % Affiliation \\
  % Address \\
  % \texttt{email} \\
  % \And
  % Coauthor \\
  % Affiliation \\
  % Address \\
  % \texttt{email} \\
  % \And
  % Coauthor \\
  % Affiliation \\
  % Address \\
  % \texttt{email} \\
}

\begin{document}
% \nipsfinalcopy is no longer used

\maketitle

\begin{abstract}
  Данная работа включает в себя результаты попытки воспроизведения статьи 
  "Generative Modeling by Estimating Gradients of the Data Distribution". Финальной целью является получение схожих со статьей результатов. На данный момент работа содержит проверку начальных экспериментов статьи.
  
  Работа устроена следующим образом, для начала расссмотрим статью, которая является основой для этой работы, далее рассмотрим нашу реализацию определенных частей статьи, подведем результаты о возможности аопсроизведения каждой из частей.
\end{abstract}


\section{Обзор статьи}

В обозреваемой статье предлагается вариант генеративной модели, использующей динамику Ланжевена основанную на оценке градиентов распределения данных с помощью score matching. В качестве архитектуры в модели используется модель RefineNet из ~\cite{DBLP:journals/corr/LinMS016} с модификацией функции активации (ELU), 

\section{Модель}

\section{Лосс}

\section{Эксперимент на смеси гауссовских распределений}

\subsection{Toy Experiment №1}
Повторим эксперимент из статьи, чтобы показать, что ssm-лосс адекватно оценивает градиент. В качестве модели был использован MLP (Multi-Layer Perceptron) с тремя слоями с размером скрытого 128 и с функцией активации Softplus. Данные генерировались из смеси распределений 
$\frac{1}{5} \mathcal{N}\left( \left( -5, -5\right), I \right)+\frac{4}{5} \mathcal{N}\left( \left( 5, 5\right), I \right)$. В качестве оптимизатора был взят Adam с lr = 0.001 с размером батча 128. 

\begin{figure}[H]
  \centering
  	\includegraphics[scale=0.5]{fig2a}
  	\caption{$\Delta_x \log p_{data}$}
  	\label{fig:2a}
\end{figure}

\begin{figure}[H]
	\centering
	\includegraphics[scale=0.5]{fig2b}
	\caption{MNIST}
	\label{fig:2b}
\end{figure}

\subsection{Toy Experiment №2}










\subsection{Figures}

%\begin{figure}
%  \centering
%  \fbox{\rule[-.5cm]{0cm}{4cm} \rule[-.5cm]{4cm}{0cm}}
%  \caption{Sample figure caption.}
%\end{figure}

All artwork must be neat, clean, and legible. Lines should be dark enough for
purposes of reproduction. The figure number and caption always appear after the
figure. Place one line space before the figure caption and one line space after
the figure. The figure caption should be lower case (except for first word and
proper nouns); figures are numbered consecutively.

You may use color figures.  However, it is best for the figure captions and the
paper body to be legible if the paper is printed in either black/white or in
color.


%\section*{References}

\bibliographystyle{unsrt} 
\bibliography{rev} 

\end{document}
